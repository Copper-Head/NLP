
\documentclass{article} %declare document class or type
\usepackage{amsmath} %load standard math symbols 
\usepackage{stmaryrd} %load additional math symbols 
\usepackage{qtree} %load support for drawing syntactic trees 
\usepackage{enumerate} %load support for custom lists 

\addtolength{\hoffset}{-2cm}
\addtolength{\textwidth}{2cm}
\addtolength{\voffset}{-2cm}
\addtolength{\textheight}{2cm}
\setlength{\parindent}{0pt}

\title{University of Massachusetts Amherst \\ COMPSCI 585: Homework 1 \\
Instructor: Andrew McCallum}
\date{\today}

\author{Sam Baldwin \\ Ilia Kurenkov}

\begin{document}
\maketitle

\section{Motivation:}

Ilia had wanted to build a chatbot for a long time. Narrowing its scope down to
something workable was always the problem, as well as figuring out how to deal
with input and output. By some stroke of luck he came upon the idea of a
secretary chatbot that would talk users into setting more or less useful dates. 

Sam found the chatbot idea sufficiently cool to back the project with their
master coding skills. And so boord4u was born.


\section{The implementation process:}
We divided the work amongst ourselves somewhat spontaneously (see section 4
Division of Labor), which might have proven to be not so good a decision
because Sam ended up having a bit of a bad time combining all of the code due
to overreaching goals of building a general-purpose finite-state machine 
simulator, which didn't end up solving any particular problem related to this
project.


\section{What next?}
If we had more time on our hands, here is what we would like to add and perfect
in the program. We are aware that most likely not all of these ideas are in
fact possible to put into practice (in fact most of them might be quite weird),
but we like our daydreaming.

\subsection{}
More modular bot responses. The final goal would be something like a rule-based
grammar that can abstractly specify valid output strings which are in turn
assembled on the fly by the bot in response to a certain input from the user.
This would, potentially, lead to a widening of the domain of the bot's
conversational competency. Perhaps one could even one day chat with the bot
about things not directly related to scheduling appointments.

\subsection{}
On a more practical level, we would like to add the possibility for the user to
exit the conversation at any time in it without having to quit the program. We
foresee the use of such expressions as "I gotta go now, we'll set the meeting
up later" or "Let me double-check with Bob and get back to you on the exact
time of that party" or even something like "I'm really not in the mood to do
this anymore, let's try another time" to somehow "understood" by the bot to
mean a request to stop in its tracks and exit.

\subsection{}
Sam is particularly interested in updating her FSA/FSM to actually handle
complicated (non-trivial) state-switching logic and robust handling of output.
This would make the task of specifying the behaviour of the bot much simpler.

\subsection{}
Alternatively, an overhaul of the system written with an eye towards functional
programming is considered at least by Sam to be potentially of merit and
towards the goal of simplicity.

\subsection{}
We would naturally be interested in having several people other than us mess
around with the program so as to be more effective at collecting bugs and other
strange behavior.

\subsection{}
On a purely practical note, we should also enable our robot to set the place of
the event as well as the time.

\subsection{}
Yet another practical concern is that of parsing time input. Currently we do
not have a very universal or robust way to do so.

\subsection{}
This is very long-term, but we would like to add support for the following
things: \\
- keeping track of current time, so as to be able to parse input like "next
week", "in two hours" or "in half a year" \\
- having a database of previously created appointments that could be retrieved
or referenced by the bot\\

\section{Division of Labor:}
\begin{enumerate}
\item
Sam - development of the implementation, most of the code including the 
Conversation class, managing input output, piecing all of the methods and 
classes into one coherent program
\item
Ilia - concept, write-up, pos\_or\_neg method, Event class, regex dictionaries
\end{enumerate}

\end{document}
